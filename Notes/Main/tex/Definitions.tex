\chapter{Definitions}
\section{Vector Spaces and Fields}
\subsection{Fields}
\begin{mydef}
\normalfont Let \(X,Y\) be sets. The \underline{Cartesian} product of \(X\) and \(Y\) is the set \(X \times Y = \{(x, y) : x \in X, y \in Y\}\)
\end{mydef}

\begin{mydef}
\normalfont Let \(f, g : X \to Y\), then \(f = g \Leftrightarrow f(x) = g(x), \forall x \in X\)
\end{mydef}

\begin{mydef}
\normalfont A \underline{binary operation} on a set \(X\) is a function from \(X \times X\) to \(X\).
\end{mydef}

\begin{mydef}
\normalfont A \underline{field} is a set \(\field\) equipped with two binary operations: 
\begin{itemize}
	\item \(+: \field \times \field \to \field\)
	\item \(\cdot: \field \times \field \to \field\)
\end{itemize}
%
satisfying the following axioms:
\begin{equation}\forall a,b,c \in \field : (a + b) + c = a + (b + c) \tag{F1}\end{equation}
\begin{equation}\forall a,b \in \field : a + b = b + a \tag{F2} \end{equation}
\begin{equation}\exists 0 \in \field : \forall a \in \field : a + 0 = a \tag{F3}\end{equation}
\begin{equation}\forall a \in \field, \exists(-a)\in\field:a+(-a)=0 \tag{F4}\end{equation}
%
\begin{equation}\forall a,b,c\in\field : a\cdot(b \cdot c) = (a \cdot b) \cdot c \tag{F5}\end{equation}
\begin{equation}\forall a,b\in\field : a \cdot b = b \cdot a \tag{F6}\end{equation}
\begin{equation}\exists 1\in\field : 1 \ne 0 \text{ and } \forall a\in\field : 1\cdot a = a \tag{F7}\end{equation}
\begin{equation}\forall a\in\field : a\ne 0, \exists a^{-1} \in \field : a\cdot a^{-1} = 1 \tag{F8}\end{equation}
%
\begin{equation}\forall a,b,c\in\field:a\cdot(b+c) = (a\cdot b) + (a\cdot c) \tag{F9}\end{equation}

\textit{Notice that F1 and F5 demonstrate associativity; F2 and F6---commutitivity; F3 and F7---identity existence; F4 and F8---inverse existence; and F9 demonstrates distributivity with respect to addition}.
\end{mydef}

\begin{mythm}\normalfont
Let \(\field\) be a field. Then
\begin{enumerate}
	\item the element 0 satisfying property (F3) is unique.
	\item for each \(a \in \field\), the element \((-a)\) satisfying (F4) is unique.
	\item the element 1 satisfying (F7) is unique.
	\item for each \(a\in\field : a \ne 0\), the element \(a^{-1}\) satisfying (F8) is unique.
\end{enumerate}
\end{mythm}

\subsection{Vector Spaces}
\begin{mydef}\normalfont
Let \(\field\) be a field. A vector space over \(\field\) is a set \(V\) together with two operations: \(+: V\times V \to V\) and \(\cdot:\field\times V\to V\) satisfying the following axioms:
%
\begin{equation}\forall u,v,w\in V : u+(v+w)=(u+v)+w\tag{VS1}\end{equation}
\begin{equation}\forall u,v\in V : u+v=v+u\tag{VS2}\end{equation}
\begin{equation}\exists0\in V \text{ s.t. } \forall u\in V : u + 0 = u \tag{VS3}\end{equation}
\begin{equation}\forall u\in V, \exists(-u)\in V : u+(-u) = 0 \tag{VS4}\end{equation}
\begin{equation}\forall a,b \in \field, u \in V : a(bu) = (ab)u \tag{VS5}\end{equation}
\begin{equation}\forall a\in\field, \forall u,v\in V : a(u+v) = au+av \tag{VS6}\end{equation}
\begin{equation}\forall a,\in\field, \forall u\in V : (a+b)u = au+bu \tag{VS7}\end{equation}
\begin{equation}\forall u\in V : 1\cdot u=u \tag{VS8}\end{equation}

\textit{Notice that VS1 and VS5 refer to associativity; VS2 to commutivity; VS3 to and VS8 to identity; VS4 to inverse; and VS6 and VS7 to distributivity}.
\end{mydef}

\begin{mythm} \normalfont
Let \(V\) be a vector space over a field \(\field\). Then 
\begin{enumerate}
	\item the element 0 satisfying (VS3) is unique
	\item \(\forall u\in V\), the element \((-u)\) satisfying (VS4) is unique lol.
\end{enumerate}
\end{mythm}


\subsection{Examples of Vector Spaces}
\subsubsection{Function Spaces}
Let \(X\) be any set, and \(\field\) be any field. The set \(V = \operatorname{Fun}(X, \field)\) is defined as the set of all functions from \(X\) to \(\field\). We equip \(V\) with the following operations of addition and scalar multiplication.

Let \(f,g\in V, \lambda \in \field\). 
Then we define \(f + g : X \to \field\) by setting
\[(f+g)(x)=f(x) + g(x), \forall x\in X\] 
and \(\lambda f : X \to \field\) by setting
\((\lambda f)(x) = \lambda(f(x))\). Thus \(f+g \in V, \lambda f \in V\).

With these operations, \(V\) is a vector space over \(\field\).

Let us prove (VS2) (commutativity of addition).
\begin{proof}
Let \(f,g\in V\). We'd like to prove that \(f+g=g+f\). Observe that 
\begin{align*}
	\forall x\in X, (f+g)(x) 	&= f(x)+g(x) \\
					&= g(x)+f(x)\text{, by (F2)} \\
					&= (g+f)(x)
\end{align*}
\(\therefore f+g=g+f\).
\end{proof}

Let us prove (VS3) (existence of 0).
\begin{proof}
We need to find some \(g\in V \text{ s.t. } \forall f\in V : f + g = f\).
Define \(g\in V\) by setting \(g(x)=0, \forall x \in X\). We can safely define \(g\) in this way because \(0\in\field\). Then \((f+g)(x)=f(x)+g(x)=f(x)+0=f(x), \forall x\in X\) by (F3).

\(\therefore f + g = f\)
\end{proof}

\subsubsection{Continuous Functions}
If \(X\) is a subset of \(\real\), then \(C(X)\) denotes the set of all continuous functions \(f:X\to\real\). Then \(C(X)\) is a vector space over \(\real\) with the same kind of operations as \(\operatorname{Fun}(X, \field)\). E.g. if \(X=\real\), then the function given by \(f(x)=\sin(5x)-3\cos(7x), \forall x \in \real\), is an element of \(C(X) : f\in C(X)\).

\subsubsection{Polynomials}
A polynomial with coefficients in \(\field\) in a variable \(t\) is a formal expression of the form
\[a_n t^n + a_{n-1}t^{n-1} + \dotsb + a_1 t^1 + a_0 t^0\]
where \(a_n, \dotsc, a_0 \in \field\). We refer to \(a_k\) as the coefficient of \(t^k\) in \(f(t)\). For example, \(3t^3 - 5t^2 + (1/7)t + \sqrt{2}\) is a polynomial with coefficients in \(\real\). When we say that this is a 'formal expression', we mean that \(t\) is a formal variable. In particular, we are \textbf{not} substituting a number (say) for \(t\). 

The usual conventions are used. The order in which the terms of a polynomial are written is immaterial. Also, adding \(0\cdot t^m\) terms does not change the polynomial. 

\begin{mydef}\normalfont
\(P_\infty = P_\infty(\field)\) denotes the set of all polynomials with coefficients in \(\field\).
\end{mydef}

We denote by . This is a vector space. Addition and multiplication by scalars are defined term-by-term, which is the usual way:
\[(a_n t^n+\dotsb+a_1 t+a_0) + (b_n t^n+\dotsb b_1 t + b_0) = (a_n + b_n)t^n+\dotsb+(a_1+b_1)t+(a_0+b_0)\]
\[\lambda(a_nt^n+\dotsb+a_1t+a_0) = (\lambda a_n)t^n+\dotsb+(\lambda a_1)t+(\lambda a_0)\]

\begin{mydef}[Degree of a polynomial]\normalfont
The degree of a polynomial \(a_n t^n+\dotsb+a_1t+a_0\) as the largest number \(d\) such that \(a_d\ne 0\). 
\end{mydef}

\(P_\infty(\field)\) becomes a vector space over \(\field\) with operations
\[\left(\sum_{k=0}^n{a_kt^k}\right) + \left(\sum_{k=0}^n b_kt^k\right) = \sum_{k=0}^n(a_k+b_k)t^k\]
%
and
%
\[\lambda\left(\sum_{k=0}^n a_kt^k\right) = \sum_{k=0}^n(\lambda a_k)t^k\]

\begin{mydef}
\(P_n(\field) = \{f\in P_\infty(\field) : \operatorname{deg}(f) \le n\}\)
\end{mydef}

\subsection{Basic Properties of a Vector Space}
\begin{mythm}[Cancellation in sums]\normalfont
If \(u,v,w\in V\) and \(u+v=u+w\), then \(v=w\).
\end{mythm}

\begin{proof}
Since \(u+v=u+w\), we have \((-u)+(u+v)=(-u)+(u+w)\). By (VS1), this implies \(((-u)+u)+v=((-u)+u)+w \Leftrightarrow 0+v=0+w\) by definition of \((-u)\). Now \(0+v=v\) and \(0+w=w\) by definition of \(0\). So \(v=w\).
\end{proof}

\begin{mythm}\normalfont 
	Consider a vector space \(V\).
	\begin{enumerate}
		\item For all \(u\in V, 0u=\underline{0}\).
		\item For all \(a\in \field, a0=\underline{0}\).
		\item For \(a\in\field\) and \(u\in V\), \((-a)u=-(au)=a(-u)\). In particular, \(-u=(-1)u\).
	\end{enumerate}
\end{mythm}

\begin{mythm}\normalfont
Consider a vector space \(V\).
\begin{enumerate}
	\item \(\forall a\in\field\) and \(u\in V\), if \(au=\underline{0}\), then either \(a=0\) or \(u=0\).
	\item (Cancellation in products)
		\begin{enumerate}
			\item For \(0\ne a\in\field\) and \(u,v\in V\), if \(au=av\) then \(u=v\).
			\item For \(a,b\in\field\) and \(0\ne u\in V\), if \(au=bu\) then \(a=b\).
		\end{enumerate}
\end{enumerate}
\end{mythm}

\begin{proof}
\begin{enumerate}
	\item If \(a=0\), there is nothing to prove, so assume \(a\ne0\). Since \(au=0\) and \(a\ne0\), we have \(a^{-1}(au)=1u=u\) by the axioms of fields and vector spaces. Hence, \(u=0\), as required.
	\item \begin{enumerate}
		\item  We have \(au=av \Leftrightarrow a^{-1}(au)=a^{-1}(av) \Leftrightarrow 1\cdot u = 1\cdot v \Leftrightarrow u=v\)
		\item Since \(au=bu\), we have \(0=au+(-bu)=au+(-b)u=(a-b)u\), where the second equality uses the previous theorem and the other equalities use the axioms of vector spaces. Now the equality \((a-b)u=0\) implies that either \(a-b=0\) or \(u=0\) by (i). We are given that \(u \ne 0\), so \(a-b=0\), whence \(a=b\).
	\end{enumerate}
\end{enumerate}
\end{proof}

\begin{mythm}\normalfont
Consider a vector space \(V\).
\begin{enumerate}
	\item We have \(a(u_1+\dotsb+u_n) = au_1 + \dotsb+au_n\) if \(a\in\field\) and \(u_1,\dotsc,u_n\in V\).
	\item We have \(a_1 + \dotsb + a_n)u=a_1 u + \dotsb + a_n u\) if \(a_1,\dotsc,a_n\in\field\) and \(u\in V\).
\end{enumerate}
\end{mythm}

\section{Subspaces}




