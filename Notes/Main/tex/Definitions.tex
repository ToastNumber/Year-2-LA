\chapter{Definitions}

\begin{mydef}
\normalfont Let \(X,Y\) be sets. The \underline{Cartesian} product of \(X\) and \(Y\) is the set \(X \times Y = \{(x, y) : x \in X, y \in Y\}\)
\end{mydef}

\begin{mydef}
\normalfont Let \(f, g : X \to Y\), then \(f = g \Leftrightarrow f(x) = g(x), \forall x \in X\)
\end{mydef}

\begin{mydef}
\normalfont A \underline{binary operation} on a set \(X\) is a function from \(X \times X\) to \(X\).
\end{mydef}

\section{Fields}
\begin{mydef}
\normalfont A \underline{field} is a set \(\field\) equipped with two binary operations: 
\begin{itemize}
	\item \(+: \field \times \field \to \field\)
	\item \(\cdot: \field \times \field \to \field\)
\end{itemize}
%
satisfying the following axioms:
\begin{equation}\forall a,b,c \in \field : (a + b) + c = a + (b + c) \tag{F1}\end{equation}
\begin{equation}\forall a,b \in \field : a + b = b + a \tag{F2} \end{equation}
\begin{equation}\exists 0 \in \field : \forall a \in \field : a + 0 = a \tag{F3}\end{equation}
\begin{equation}\forall a \in \field, \exists(-a)\in\field:a+(-a)=0 \tag{F4}\end{equation}
%
\begin{equation}\forall a,b,c\in\field : a\cdot(b \cdot c) = (a \cdot b) \cdot c \tag{F5}\end{equation}
\begin{equation}\forall a,b\in\field : a \cdot b = b \cdot a \tag{F6}\end{equation}
\begin{equation}\exists 1\in\field : 1 \ne 0 \text{ and } \forall a\in\field : 1\cdot a = a \tag{F7}\end{equation}
\begin{equation}\forall a\in\field : a\ne 0, \exists a^{-1} \in \field : a\cdot a^{-1} = 1 \tag{F8}\end{equation}
%
\begin{equation}\forall a,b,c\in\field:a\cdot(b+c) = (a\cdot b) + (a\cdot c) \tag{F9}\end{equation}

\textit{Notice that F1 and F5 demonstrate associativity; F2 and F6---commutitivity; F3 and F7---identity existence; F4 and F8---inverse existence; and F9 demonstrates distributivity with respect to addition}.
\end{mydef}

\begin{mythm}\normalfont
Let \(\field\) be a field. Then
\begin{enumerate}
	\item the element 0 satisfying property (F3) is unique.
	\item for each \(a \in \field\), the element \((-a)\) satisfying (F4) is unique.
	\item the element 1 satisfying (F7) is unique.
	\item for each \(a\in\field : a \ne 0\), the element \(a^{-1}\) satisfying (F8) is unique.
\end{enumerate}
\end{mythm}

\section{Vector Spaces}
\begin{mydef}\normalfont
Let \(\field\) be a field. A vector space over \(\field\) is a set \(V\) together with two operations: \(+: V\times V \to V\) and \(\cdot:\field\times V\to V\) satisfying the following axioms:
%
\begin{equation}\forall u,v,w\in V : u+(v+w)=(u+v)+w\tag{VS1}\end{equation}
\begin{equation}\forall u,v\in V : u+v=v+u\tag{VS2}\end{equation}
\begin{equation}\exists0\in V \text{ s.t. } \forall u\in V : u + 0 = u\tag{VS3}\end{equation}
\begin{equation}\forall u\in V, \exists(-u)\in V : u+(-u) = 0\tag{VS4}\end{equation}
\begin{equation}\forall a,b \in \field, u \in V : a(bu) = (ab)u\tag{VS5}\end{equation}
\begin{equation}\forall a\in\field, \forall u,v\in V : a(u+v) = au+av\tag{VS6}\end{equation}
\begin{equation}\forall a,\in\field, \forall u\in V : (a+b)u = au+bu\tag{VS7}\end{equation}
\begin{equation}\forall u\in V : 1\cdot u=u\tag{VS8}\end{equation}

\textit{Notice that VS1 and VS5 refer to associativity; VS2 to commutivity; VS3 to and VS8 to identity; VS4 to inverse; and VS6 and VS7 to distributivity}.
\end{mydef}

\begin{mythm} \normalfont
Let \(V\) be a vector space over a field \(\field\). Then 
\begin{enumerate}
	\item the element 0 satisfying (VS3) is unique
	\item \(\forall u\in V\), the element \((-u)\) satisfying (VS4) is unique.
\end{enumerate}
\end{mythm}


